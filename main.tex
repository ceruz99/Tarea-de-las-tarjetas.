\documentclass{article}
\usepackage[utf8]{inputenc}
\usepackage[spanish]{babel}
\usepackage{listings}
\usepackage{graphicx}
\graphicspath{ {images/} }
\usepackage{cite}

\begin{document}

\begin{titlepage}
    \begin{center}
        \vspace*{1cm}
            
        \Huge
        \textbf{Pirámide con tarjetas}
            
        \vspace{1cm}
        \LARGE
    
            
        \vspace{1.5cm}
            
        \textbf{Juan Camilo Mazo Castro}
            
        \vfill
            
        \vspace{0.8cm}
            
        \Large
        Despartamento de Ingeniería Electrónica y Telecomunicaciones\\
        Universidad de Antioquia\\
        Medellín\\
        Marzo de 2021
            
    \end{center}
\end{titlepage}

\tableofcontents
\newpage
\section{Sección introductoria}\label{intro}
El presente trabajo se realizó con el fin de hacer una prepación previa relacionada con el manejo de herramientas para edición de videos para el examen parcial. Además de reflexionar en la importancia de dar instrucciones totalmente claras que no dejen cabida para la ambigüedad. 

\section{Sección de contenido} \label{contenido}
1. Tenga en cuenta que todos los siguientes pasos se realizarán con una sola mano.\\\\
2. Levante la hoja de papel y póngala a un lado sobre la superficie horizontal de la mesa.\\\\
3. Tome ambas tarjetas con una mano.\\\\
4. Mantenga las tarjetas agarradas, toma la hoja de papel y la pone en su lugar inicial donde estaba con las tarjetas pero esta vez sin colocar las tarjetas, únicamente la hoja.\\\\
5. Con las tarjetas agarradas juntarlas de manera completamente simétrica.\\\\
6. Observe que las tarjetas tienen dos caras, dos lados largos y dos lados más cortos.\\\\
7. Ponga el dedo pulgar en uno de los lados largos de las tarjetas.\\\\
8. Ponga el dedo índice en el lado superior de tamaño más corto de las tarjetas.\\\\
9. Ponga el resto de dedos en el lado opuesto al lado en el que tiene el pulgar, no ponga ninguno de los dedos sobre las caras.\\\\
10. El único lado que quedó sin ningún dedo lo va a apoyar sobre la hoja de papel sin soltar las tarjetas y manteniendo la posiciones de los dedos en las tarjetas como se indicó anteriormente.\\\\
11. Sin quitar los dedos de donde están y manteniendo juntos el lado de las tarjetas donde está el índice, con la ayuda de la superficie de la hoja y apoyado las tarjetas en esta, intente separar los lados que están sobre la superficie de la hoja.\\\\
12. separar los lados forme una pirámide sin separar los lados donde está el índice.\\\\
13. Una vez esté estable la pirámide suelte las tarjetas.


\bibliographystyle{IEEEtran}

\end{document}
